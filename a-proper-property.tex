% header

\documentclass{beamer}
\usepackage[english]{babel}
\usepackage{amsmath,amssymb}
\usepackage[utf8]{inputenc}
\usetheme{Berkeley}

\usepackage{color}
\usepackage{xcolor}
\usepackage{listings}
\usepackage{url}
\usepackage[T1]{fontenc}
\usepackage{xspace}
\usepackage[style=iso]{datetime2}

\usepackage{tikz}
\usetikzlibrary{shapes.callouts,shadows, calc}
\usepackage{listings}


% Solarized colour scheme for listings
%
\definecolor{solarized@base03}{HTML}{002B36}
\definecolor{solarized@base02}{HTML}{073642}
\definecolor{solarized@base01}{HTML}{586e75}
\definecolor{solarized@base00}{HTML}{657b83}
\definecolor{solarized@base0}{HTML}{839496}
\definecolor{solarized@base1}{HTML}{93a1a1}
\definecolor{solarized@base2}{HTML}{EEE8D5}
\definecolor{solarized@base3}{HTML}{FDF6E3}
\definecolor{solarized@yellow}{HTML}{B58900}
\definecolor{solarized@orange}{HTML}{CB4B16}
\definecolor{solarized@red}{HTML}{DC322F}
\definecolor{solarized@magenta}{HTML}{D33682}
\definecolor{solarized@violet}{HTML}{6C71C4}
\definecolor{solarized@blue}{HTML}{268BD2}
\definecolor{solarized@cyan}{HTML}{2AA198}
\definecolor{solarized@green}{HTML}{859900}

% tikz listings support for highlighting - just insert backticks.
\tikzset{note/.style={rectangle callout, rounded corners,fill=gray!20,drop shadow,font=\footnotesize}}    

\newcommand{\tikzmark}[1]{\tikz[overlay,remember picture] \node (#1) {};}    

\newcounter{calloutcounter}
\setcounter{calloutcounter}{1}

\makeatletter
\newenvironment{btHighlight}[1][]
{\begingroup\tikzset{bt@Highlight@par/.style={#1}}\begin{lrbox}{\@tempboxa}}
{\end{lrbox}\bt@HL@box[bt@Highlight@par]{\@tempboxa}\endgroup}

\newcommand\btHL[1][]{%
  \begin{btHighlight}[#1]\bgroup\aftergroup\bt@HL@endenv%
}
\def\bt@HL@endenv{%
  \end{btHighlight}%   
  \egroup
}
\newcommand{\bt@HL@box}[2][]{%
  \tikz[#1]{%
    \pgfpathrectangle{\pgfpoint{0pt}{0pt}}{\pgfpoint{\wd #2}{\ht #2}}%
    \pgfusepath{use as bounding box}%
    \node[anchor=base west,rounded corners, fill=green!30,outer sep=0pt,inner xsep=0.2em, inner ysep=0.1em,  #1](a\thecalloutcounter){\usebox{#2}};
  }%
   %\tikzmark{a\thecalloutcounter} <= can be used, but it leads to a spacing problem
   % the best approach is to name the previous node with (a\thecalloutcounter)
 \stepcounter{calloutcounter}
}
\makeatother


\lstset{
  language=C++,
  basicstyle=\footnotesize\ttfamily,
  rangeprefix=//\ ,
  includerangemarker=false,
}

% Define C++ syntax highlighting colour scheme
\lstdefinelanguage{cpp}{
  language=C++,
  basicstyle=\footnotesize\ttfamily,
  numbers=left,
  numberstyle=\footnotesize,
  tabsize=2,
  breaklines=true,
  escapeinside={@}{@},
  numberstyle=\tiny\color{solarized@base01},
  keywordstyle=\color{solarized@green},
  stringstyle=\color{solarized@cyan}\ttfamily,
  identifierstyle=\color{solarized@blue},
  commentstyle=\color{solarized@base01},
  emphstyle=\color{solarized@red},
  frame=single,
  rulecolor=\color{solarized@base2},
  rulesepcolor=\color{solarized@base2},
  showstringspaces=false,
  moredelim={**[is][\btHL]{`}{`}},
}

\lstdefinelanguage{diff}{
  morecomment=[f][\color{blue}]{@@},           % group identifier
  morecomment=[f][\color{red}]{-},             % deleted lines
  morecomment=[f][\color{green!50!black}]{+},  % added lines
  morecomment=[f][\color{magenta}]{---},       % diff header lines
  morecomment=[f][\color{magenta}]{+++},
}

\lstdefinelanguage{plus}{
  basicstyle=\footnotesize\ttfamily\color{green!50!black},
  emph={see,below,TypeSwitch,unspecified},
  emphstyle=\itshape
}

\lstdefinelanguage{signature}{
  basicstyle=\ttfamily\color{green!50!black},
  emph={see,below,TypeSwitch,unspecified},
  emphstyle=\itshape
}

\newcommand{\desc}[1]{\textit{#1}}
\newcommand{\requires}{\desc{Requires}}
\newcommand{\effects}{\desc{Effects}}
\newcommand{\precondition}{\desc{Precondition}}
\newcommand{\postcondition}{\desc{Postcondition}}
\newcommand{\throws}{\desc{Throws}}
\newcommand{\returns}{\desc{Returns}}
\newcommand{\remarks}{\desc{Remarks}}
\newcommand{\exceptionsafety}{\desc{Exception Safety}}
\newcommand{\fullref}[1]{\ref{#1} \nameref{#1}}

\def\code#1{\texttt{#1}}
\newcommand\mypound{\protect\scalebox{0.8}{\raisebox{0.4ex}{\#}}}
\newcommand{\CC}{C\nolinebreak\hspace{-.05em}\raisebox{0.4ex}{\resizebox{!}{0.6\baselineskip}{\bf++}}}
\newcommand{\cplusplus}{\protect\CC\xspace}
\newcommand{\this}{\code{this}\xspace}
\newcommand{\self}{\code{self}\xspace}
\newcommand{\cvref}{\emph{cv-ref qualifiers}\xspace}
\newcommand{\nl}{\vspace{0.2\baselineskip}}
\newcommand{\HT}{{\Huge T}{\hspace{-5pt}}}
\newcommand{\HI}{{\Huge I}{\hspace{-1pt}}}
\newcommand{\HS}{{\Huge S}{\hspace{-1pt}}}
\newcommand{\HK}{{\Huge K}{\hspace{-1pt}}}

\title{A Proper Property}
\author{Gašper Ažman}
\date{November 27, 2017}


\begin{document}

\begin{frame}
  \titlepage
\end{frame}


\section{Author}
\begin{frame}
  \frametitle{About Me}
  \begin{itemize}
    \item Gašper Ažman (twitter: @atomgalaxy)
    \item Started teaching \cplusplus in highschool
    \item Did computer vision research at Berkeley
    \item Helped with Amazon Search Infrastructure at a9.com
    \item Currently at Citadel, building really cool research tools.
    \item A regular at the British Standards Insitute (BSI) Meetings
    \item On the programming committes of CppCon and C$++$Now.
  \end{itemize}
\end{frame}

\section{Disclaimer}

\begin{frame}
\frametitle{Disclaimer}
The opinions in this talk are my own, and do not necessarily reflect the
opinions of Citadel LLC or any of its subsidiaries.\nl

In addition, no Citadel resources were used in the development of this talk.
\end{frame}


\section{Recap}
\begin{frame}[fragile]

\frametitle{So, What is a Property?}
\begin{center}
{\Large A property pretends to be a field.}
\end{center}

Assignment:
\begin{lstlisting}[language=cpp]
airplane.cargo = "hot air";
\end{lstlisting}

Reading:
\begin{lstlisting}[language=cpp]
Payload x = airplane.cargo;
\end{lstlisting}

(Aside: we need a payload, and strings do nicely.)
\begin{lstlisting}[language=cpp]
// books truly are the greatest gift
using Payload = std::string;
using MaybePayload = std::optional<Payload>
\end{lstlisting}

\end{frame}


\begin{frame}[fragile]
\frametitle{Totally.}

\begin{center}
If it quacks like a field, it has to be...
\end{center}

\begin{lstlisting}[language=cpp]
class Airplane {
  MaybePayload cargo;
} airplane;
\end{lstlisting}

\begin{center}
... a field, right?
\end{center}
\end{frame}


\begin{frame}[fragile]
\frametitle{Have you heard of this?}

\includegraphics[width=\textwidth]{CargoCultAirplane.jpg}
\end{frame}


\begin{frame}[fragile]
\frametitle{It is only a shell...}

\begin{center}
Instead, it's a pair of a Getter and a Setter on a member.
\end{center}
\begin{lstlisting}[language=cpp]
class Airplane {
  struct Cargo {
    // Setter - assignment from Payload
    void operator=(Payload crate) {
      return {};
    }
    // Getter - implicit conversion to Payload
    operator Payload() const {
      return {};
    }
  };
  Cargo cargo;
} airplane;
\end{lstlisting}
\begin{center}
{\small(We need at least one byte to give \code{cargo} an address)}.
\end{center}
\end{frame}

\section{The Problem}

\begin{frame}[fragile]
  \frametitle{Let's Get More Interesting}
\begin{lstlisting}[language=cpp]
class Airplane {
  MaybePayload `hold`;
public: @\pause@
  struct Cargo {
    MaybePayload operator=(Payload crate) {
      if (`hold`) return {std::move(crate)};
      @@`hold` = std::move(crate); return {};
    }@\pause@
    operator Payload() {
      if (!`hold`) throw NoPayloadError{};
      return *std::exchange(`hold`, {});
    } @\pause@
  } cargo;
} airplane;
\end{lstlisting}
\pause
Pro: this solution is pretty.
\pause

Con: it is not a solution. {\tiny (It does not compile.)}
\end{frame}

\begin{frame}
  \frametitle{But WHY?}
\begin{center}
  {\Huge We need (Airplane*) \this}
  \vspace{.8 in} 

  Not \code{(Cargo*) this}.
\end{center}
\end{frame}


\begin{frame}[fragile]
  \frametitle{But... We wants it! We needs it, precious!}

\begin{center}
  {\Huge No.}
  \vspace{.8 in} 

  You're not getting a second breakfast... I mean, a second \this.
\end{center}
\end{frame}


\begin{frame}
  \frametitle{The Problem}

\begin{center}
  {\Huge Get The Host's \this.} \nl

  ... while being reasonably easy to use.
\end{center}
\end{frame}


\section{KISS}
\begin{frame}
\begin{center}
  Attempt 1:\nl\nl

  \HK eep \HI t \HS urprisingly \HS imple
\end{center}
\end{frame}


\begin{frame}[fragile]
\frametitle{Store the \this pointer}
\begin{lstlisting}[language=cpp]
class Airplane {
  MaybePayload `hold`;
public:
  struct Cargo {
    MaybePayload operator=(Payload crate) {
      if (`host`->hold) return {std::move(crate)};
      @@`hold` = std::move(crate); return {};
    }
    operator Payload() {
      if (!`host`->hold) throw NoPayloadError{};
      return *std::exchange(`hold`, {});
    }
    @@`Airplane* const host`;
  } cargo;
  @@`Airplane() : cargo{this} {}` // every. time.
} airplane;
\end{lstlisting}
\end{frame}


\begin{frame}[fragile]
  \frametitle{So... How'd we do?}
\begin{center}
{\Huge Poorly.}
\end{center}
\begin{description}
  \item[Needs extra space?] Check.
  \item[Error-prone?] Check.
  \item[No help from C++?] Check.
  \item[Easy?] To understand, yes. To maintain? Good luck.
    (does not pass the WWTDCD\footnote{What Would The Default Constructor Do}
    test)
\end{description}
\end{frame}


\section{Solution Criteria}
\begin{frame}
  \frametitle{Moving the Goalposts Much?}
\begin{center}
{\Huge We need some criteria.}
\end{center}
\end{frame}


\begin{frame}[fragile]
  \frametitle{Criteria}
  \textbf{The First Rule of \cplusplus}\pause

  We only pay for what we use.\pause\nl\nl\nl


  \textbf{At Most One Macro Per Property}

  The generated code must be contiguous.\pause\nl\nl\nl

  \textbf{No Pitfalls}
  \begin{itemize}
    \item Easy to read
    \item Easy to write
    \item Easy to modify
  \end{itemize}

  \pause
  \begin{center}
    Boils down to \emph{Don't repeat yourself.}
    
    And we had to repeat ourselves with every constructor and assignment
    operator.
  \end{center}
\end{frame}


\section{A Better Idea}

\begin{frame}[fragile]
\frametitle{offsetof}
\begin{center}
  Attempt 2:\nl\nl

  {\Huge offsetof}
\end{center}
\end{frame}


\begin{frame}[fragile]
\frametitle{Member Offsets are Constant}
\begin{center}
  We already have \code{(Cargo*) this}. We can \textbf{compute}
  \code{(Airplane*) this}.
  \includegraphics[width=\textwidth]{offsets.pdf}

  x86\_64, clang: \code{\&Airplane::cargo} == 32
\end{center}
\end{frame}


\begin{frame}[fragile]
\frametitle{Easy Peasy!}
\begin{center}
  We have:
\end{center}
\begin{lstlisting}[language=cpp]

  auto offset = offsetof(Airplane, cargo);
  auto fixed  =
                            this  - offset;
  auto host   =                             fixed ;
                  
 
\end{lstlisting}
\end{frame}


\begin{frame}[fragile]
\frametitle{Now For Something That Actually Works}
\begin{center}
  With the casts in:
\end{center}
\begin{lstlisting}[language=cpp]

  auto offset = offsetof(Airplane, cargo);
  auto fixed  =
    reinterpret_cast<char*>(this) - offset;
  auto host   = reinterpret_cast<Airplane*>(fixed);
                    
 
\end{lstlisting}
\end{frame}


\begin{frame}[fragile]
\frametitle{If You Like It, Put It In A Function}
\begin{center}
  With a function around it:
\end{center}
\begin{lstlisting}[language=cpp]
Airplane* get_host() {
  auto offset = offsetof(Airplane, cargo);
  auto fixed  =
    reinterpret_cast<char*>(this) - offset;
  auto host  = reinterpret_cast<Airplane*>(fixed);
  return host;
}
\end{lstlisting}
\end{frame}


\begin{frame}[fragile]
\frametitle{But...}
\begin{center}
  But, it has warnings!
\end{center}
\begin{verbatim}
offset_of.cpp:34:23: warning: offset of
  on non-standard-layout type 'Airplane'
  [-Winvalid-offsetof]

auto offset   = offsetof(Airplane, cargo);
                ^                  ~~~~~
\end{verbatim}
(No, it's not UB, if you're using c++17)
\end{frame}


\begin{frame}[fragile]
\frametitle{But, Is it... Legal?}
\begin{center}
  Turns out this doesn't have a warning, and is \code{constexpr}.
\end{center}
\begin{lstlisting}[language=cpp]
  std::integral_constant<size_t, offsetof(Airplane, cargo)>::value;
\end{lstlisting}
\end{frame}


\begin{frame}[fragile]
\frametitle{So... How'd we do?}
\begin{center}
  We did OK. For a once-off.\nl\nl

\begin{itemize}
  \item The getter and setter pair are completely ad-hoc.
  \item Ad-Hoc \code{get\_host()} function with scary casts.
\end{itemize}

\end{center}
\end{frame}

\section{Preview}
\begin{frame}[fragile]
\frametitle{How About This?}
\begin{lstlisting}[language=cpp]
class Airplane {
  MaybePayload hold;
  struct Cargo {
    template <typename Host>
    auto set(Host& `host`, Payload payload) const {
      if (`host.hold`) return {std::move(payload)};
      @@`host.hold` = std::move(payload); return {};
    }
    template <typename Host>
    auto get(Host& `host`) const {
      if (!`host.hold`) { throw NoPayloadError{}; }
      return *std::exchange(`host.hold`, {});
    }
  };
public:
  @@`LIBPROPERTY_WRAP((Cargo), cargo, Airplane);`
};
\end{lstlisting}
\end{frame}


\begin{frame}
  \frametitle{The Anticlimax}
\begin{center}
  I don't think we can get away without macros. \nl \nl \pause

  But, I promise they're not the worst thing about this solution. \nl \nl \pause

  Wait, that's not a good thing. \nl \nl \nl \nl \nl \pause

  ... well, maybe they are.
\end{center}
\end{frame}


\begin{frame}[fragile]
\frametitle{Down The Rabbit-Hole}
\begin{center}
  We had:
\end{center}
\begin{lstlisting}[language=cpp]
class Airplane {
  MaybePayload hold;
  struct Cargo {
    void operator=(Payload crate) {
      auto& host = *Airplane::`get_host(this)`;
      /* use host.hold */
    }
  };
public:
  Cargo cargo;
  static Airplane* `get_host`(Cargo* cargo) {
    return /* cargo - offsetof(Airplane, cargo); */
  }
};
\end{lstlisting}
\end{frame}


\begin{frame}[fragile]
\frametitle{But What About Second Cargo?}
\begin{center}
  This is pretty doable:
\end{center}
\begin{lstlisting}[language=cpp]
class Airplane {
  MaybePayload hold;
  MaybePayload frunk;
  template <auto managed>
  struct Cargo {
    void operator=(Payload crate) {
      auto& host = `Airplane`::get_host(*this);
      host.*managed = std::move(crate);
    }
  };
public:
  Cargo<&Airplane::hold> cargo;
  static Airplane& `get_host(decltype(cargo)&)`;
  Cargo<&Airplane::frunk> front_bay; 
  static Airplane& `get_host(decltype(front_bay)&)`;
};
\end{lstlisting}
\end{frame}


\begin{frame}[fragile]
\frametitle{OK Now?}
\begin{center}
  Better, but we still need to:
\end{center}
\begin{itemize}
  \item manually get the host pointer.
  \item We need at least 3 overloads of \code{get\_host: (const\&, \&, \&\&)}
    per member.
  \item We need to manually type \code{Airplane::}
  \item \code{get\_host()} pollutes the interface of Airplane, and choosing an
    uglier and less-likely-to-clash name makes our implementation uglier too.
\end{itemize}

\end{frame}


\begin{frame}[fragile]
\frametitle{In Other News, Stack Corruption.}
\begin{center}
  Also, what about this?
\end{center}

\begin{lstlisting}[language=cpp]
Airplane airplane;
auto x = airplane.cargo; // works!
x = "foo";               // corrupts the stack
\end{lstlisting}
\end{frame}


\begin{frame}[fragile]
\frametitle{Fix}
\begin{lstlisting}[language=cpp]
  template <auto managed>
  struct Cargo {
    @@`friend Airplane`;
    /* getters, setters */
    @@`private:`
    Cargo() = default;
    Cargo(Cargo const&) = default;
    Cargo(Cargo&&) = default;
    Cargo const& operator=(Cargo const&) = default;
    Cargo&& operator=(Cargo&&) = default;
    ~Cargo() = default;
  };
\end{lstlisting}
\begin{center}
  Now only Airplane can manage Cargo.
\end{center}
\end{frame}


\begin{frame}[fragile]
\frametitle{... Better.}
\begin{center}
  Now Breaks:
\end{center}

\begin{lstlisting}[language=cpp]
Airplane airplane;
// breaks, copy constructor is private.
auto x = airplane.cargo;
\end{lstlisting}
\end{frame}


\begin{frame}[fragile]
\frametitle{Really Though?}
\begin{center}
  This is a lot of code. We want to put this into a library.\nl \nl \nl \nl

  It gets to be \emph{a lot more code} when you want return-type deduction,
  SFINAE, and templates for getters and setters to work correctly.
\end{center}
\end{frame}

\section{SIITT}
\begin{frame}[fragile]
\frametitle{Store It In The Type}
\begin{center}
  Attempt 3:\nl\nl

  \HS tore \HI t \HI n \HT he \HT ype.
\end{center}
\end{frame}


\begin{frame}[fragile]
\frametitle{So, What do We Need?}
\begin{center}
  We need to wrap our getter/setter provider into an adaptor.
\end{center}
\begin{lstlisting}[language=cpp]
template <typename Property, typename Tag>
class wrapper {
  // allow `host` to access self::value
  friend host;
  Property value;

  constexpr wrapper()=default;
  constexpr wrapper(wrapper const&)=default;
  constexpr wrapper(wrapper&&)=default;
  ~wrapper()=default;
  constexpr wrapper& operator=(wrapper const&)=default;
  constexpr wrapper& operator=(wrapper&&)=default;
\end{lstlisting}
\end{frame}


\begin{frame}[fragile]
\frametitle{Setters}
\begin{lstlisting}[language=cpp]
template <typename Property, typename Tag>
class wrapper { /*cont*/
  /* setter implementation */
  template <typename X>
  decltype(auto) operator=(X&& val) & {
    return value.set(
        ::libproperty::impl::get_host(*this),
          std::forward<X>(val));
  }
  /* and the const& and && variants */
\end{lstlisting}
\end{frame}

\begin{frame}[fragile]
\frametitle{Getters}
\begin{lstlisting}[language=cpp]
template <typename Property, typename Tag>
class wrapper { /*cont*/
  // SFINAE-detect getter presence
  // also: you must say it 3 times (Vittorio, thanks!)
  template <typename V = value_type, // fake params
            typename H = host,
            typename = decltype(
              std::declval<V>().get(
                std::declval<`H const&`>()))>
  auto get() `const &` -> decltype(auto)
  {
    return value.get(
      ::libproperty::impl::get_host(*this));
  }
  /* and the & and && variants */
\end{lstlisting}
\end{frame}


\begin{frame}[fragile]
\frametitle{Implicit Conversions}
\begin{lstlisting}[language=cpp]
template <typename Property, typename Tag>
class wrapper { /*cont*/
  // SFINAE-detect get() presence...
  // also: you must say it 3 times (Vittorio, thanks!)
  /* and the & and && variants */
  template <typename W = wrapper> // type-dependent
  operator decltype(
    std::declval<W const&>().get())() const &
  {
    return get();
  }
\end{lstlisting}
\end{frame}


\begin{frame}[fragile]
\frametitle{The Magic Macro}
\begin{lstlisting}[language=cpp]
#define LIBPROPERTY_WRAP(type, name, host) \
  LIBPROPERTY__DECLARE_TAG(name, host);    \
  ::libproperty::wrapper<                  \
      LIBPROPERTY__PARENTHESIZED_TYPE type,\
      host::LIBPROPERTY__TAG_NAME(name)>   \
      name;                                \
  static_assert("require semicolon")

#define LIBPROPERTY__DECLARE_TAG(name, host)  \
  struct _libproperty__##name##_prop_tag {    \
    using host_type = host;                   \
    auto static constexpr offset()            \
    {                                         \
      return std::integral_constant<size_t,   \
        offsetof(host, name)>{};              \
    }                                         \
  };                                          \
  static_assert("require semicolon")
\end{lstlisting}
\end{frame}

\begin{frame}[fragile]
\frametitle{That's it!}
\begin{center}
  There are a few helpers to ferry data to-and-fro.\nl\nl\nl

  The trick is really in defining the tag type *outside* the property, so we can
  reuse our wrapper. \nl\nl\nl

  The \emph{other} trick is doing the \code{offsetof} inside an
  \code{auto}-typed constxpr function that returns an integral
  constant.\nl\nl\nl

  This defers lookup until all the types of all data members are known.
\end{center}
\end{frame}


\begin{frame}[fragile]
\frametitle{So, This Works!}
\begin{center}
  We have avoided many, many hours of "you can't do this because the type isn't
  defined yet" with this path.\nl\nl\nl

  Libproperty has a lot more cool features:
\end{center}
\begin{itemize}
  \item You can get the host pointer as a universal reference, and thus only
    have to write one getter template.
  \item The host is specified once, in the macro.
  \item You can store the value *in* the Cargo object. We could do that here
    too, and avoided the space penalty, but there are pitfalls.
  \item If you want comparisons with strings to work, you need to overload all
    of them - the library forwards those for you.
\end{itemize}
\end{frame}

\section{FIN}

\begin{frame}[fragile]
\frametitle{Questions?}
\begin{center}
  {\Huge Thank You.}
\end{center}
\end{frame}


\end{document}
